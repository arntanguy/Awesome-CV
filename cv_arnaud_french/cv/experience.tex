%-------------------------------------------------------------------------------
%	SECTION TITLE
%-------------------------------------------------------------------------------
\cvsection{Expérience}

%-------------------------------------------------------------------------------
%	CONTENT
%-------------------------------------------------------------------------------
\begin{cventries}

%---------------------------------------------------------
  \cventry
    {Ingénieur de recherche} % Job title
    {Joint Robotics Laboratory -- Advanced Institute of Science of Technology} % Organization
    {Tsukuba, Japan} % Location
    {Nov. 2019 - March. 2020} % Date(s)
    {
      \begin{cvitems} % Description(s) of tasks/responsibilities
        \item Responsable de la fusion des logiciels de controle entre:\\
          -- Le framework \emph{mc\_rtc} developpe par le CNRS par l'equipe IDH au LIRMM (Montpellier) et l'AIST-JRL (Tsukuba)\\
          -- Le framework \emph{HMC} developpe par le groupe HRG a l'AIST (Tsukuba)
        \item Soutient technique aux deux groupes, developpement de demonstrations sur les robots afin d'assurer les contributions experimentales des publications scientifiques, et de repondre aux attentes de nos partenaires industriels.
      \end{cvitems}
    }

%---------------------------------------------------------
  \cventry
    {Ingénieur de recherche} % Job title
    {Interactive Digital Human -- LIRMM} % Organization
    {Montpellier, France} % Location
    {Oct. 2018 - Oct. 2019} % Date(s)
    {
      \begin{cvitems} % Description(s) of tasks/responsibilities
        \item \entrypositionstyle{H2020 COMANOID - Multi-Contact Collaborative Humanoids in Aircraft Manufacturing}\\
          \textbf{Responable de l'implementation et integration des methodes de localisation et cartographie} pour la demonstration finale du projet europeen COMANOID. Cette demonstration est le resultat de 4 ans d'efforts partages entre 4 instituts de recherche (LIRMM, DLR, Sapienza Univeristy of Rome, INRIA Rennes). Elle a ete realisee avec succes sur le site industriel de notre partenaire industriel Airbus a permis de demontrer la viabilite d'utiliser des robots humanoides dans le cadre industriel reel de construction d'avions. Les taches effectuees visent a montrer les capacites de locomotion et manipulation dans un espace industriel representatif d'Airbus, dont: marche et localisation d'un escalier (\textbf{SLAM}), montee d'escalier (\textbf{MPC}), marche et localisation de ``brackets", saisie des ``brackets"(\textbf{visual servoing}), localisation et application a un espace du fuselage pre-determine sur le modele CAO (SLAM, registration, visual servoing).
        \item Developpement et maintinent du framework de controle `mc\_rtc` utilise lors de la demonstration ci-dessus, ainsi que par les etudiants et chercheur du LIRMM et JRL.
        \item Soutient technique aux etudiants et chercheurs et realisation d'experiences sur les robot HRP-4 et BAZAR.
      \end{cvitems}
    }



%---------------------------------------------------------
  \cventry
    {These} % Job title
    {LIRMM, I3S, JRL} % Organization
    {France, Japon} % Location
    {Oct. 2014 - Nov. 2018} % Date(s)
    {
      \begin{cvitems} % Description(s) of tasks/responsibilities
        \item \entrypositionstyle{Directeurs}: Abderrahmane Kheddar, Andrew Comport 
        \item \entrypositionstyle{Projects}: RobotHow, H2020 COMANOID, DARPA Robotics Challenge
        \item Localisation d'un robot humanoide et de son environement exploitant l'etat de l'art du SLAM Visuel Dense. Collarboration avec Dr. Andrew Comport.
        \item Localisation d'objets par registration de modeles CAO avec la carte dense du SLAM (Iterative Closest Point). Ameliorations permettant de prendre en compte des differences d'echelle entre le modele theorique et celui observe.
        \item Adaptation en ligne de plans de locomotion multi-contacts generes hors ligne exploitant les informations visuelles sus-mentionnees. 
        \item Developpement d'une methode de calibration corps-complet ne necessitant pas de marqueurs.
        \item Marche par commande predictive de modele (MPC), exploitant une fusion d'informations visuelles (SLAM) et de capteurs propioceptifs (encoders, force-sensors) permettant de reagir a des perturbations en generant continuellement une trajectoire de ZMP et de pas futur assurant la stabilite du robot.
        \item \entrypositionstyle{\emph{DARPA Robotics Challenge (DRC)}}: Pariticipation au sein de l'equipe AIST-NEDO. Classes 10/23 avec l'accompilissement de 6 des 8 taches (conduite semi-autonome, ouverture de porte et d'une vanne, percer un trou dans un mur, brancher un cable, traverser un terrain accidente). Utilisation des methodes de registration proposee dans cette these.
      \end{cvitems}
    }

%---------------------------------------------------------
  \cventry
    {Stagiaire} % Job title
    {Technische Universität München (TUM)} % Organization
    {Munich, Allemagne} % Location
    {2014 (6 months)} % Date(s)
    {
      \begin{cvitems} % Description(s) of tasks/responsibilities
        \item Superviseurs: Jurgen Sturm et Daniel Cremers
        \item Exploration de l'utilisation de reseaux de neurones convolutionels appliques a la detection de fermeture de boucle du SLAM.
        \item Developpement de l'architecture permettant l'utilisation de reseaux Siamois dans le framework open-source \emph{Caffe}.
      \end{cvitems}
    }

%---------------------------------------------------------
  \cventry
    {Projets universitaires} % Job title
    {Polytech Nice-Sophia, Trinity College Dublin} % Organization
    {Munich, Allemagne} % Location
    {2014 (6 months)} % Date(s)
    {
      \begin{cvitems} % Description(s) of tasks/responsibilities
        \item Developpement d'un moteur physique et de rendu (simulation de fluide, collisions entre objets rigide, collisions objets/fluide, raytracing).
        \item Developpement d'un logiciel de fitting interactif de courbes specialise pour la recherche en spectrospie par microscope a effet tunnel. 
        \item Rendu photo-realiste de cartes SLAM dans un Occulus Rift (projet dirige par Andrew Comport).
        \item Developpement d'un jeu de course 3D pour joueurs a handicap visuel. 
      \end{cvitems}
    }

%---------------------------------------------------------
  \cventry
    {Lyceen, projet C++ d'apprentissage auto-didacte} % Job title
    {Fotowall} % Organization
    {Brest, France} % Location
    {2008-2011} % Date(s)
    {
      \begin{cvitems} % Description(s) of tasks/responsibilities
        \item Deloppement auto-didacte d'un logiciel C++ open-source de manipulation d'image Fotowall
        \item Collaboration avec un developpeur italien
        \item Plus de 470.000 utilisateurs (en decembre 2011)
      \end{cvitems}
    }

\end{cventries}

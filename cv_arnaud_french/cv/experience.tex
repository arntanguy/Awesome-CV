%-------------------------------------------------------------------------------
%	SECTION TITLE
%-------------------------------------------------------------------------------
\cvsection{Expérience}

%-------------------------------------------------------------------------------
%	CONTENT
%-------------------------------------------------------------------------------
\begin{cventries}

%---------------------------------------------------------
  \cventry
    {Ingénieur de recherche} % Job title
    {Joint Robotics Laboratory -- Advanced Institute of Science of Technology} % Organization
    {Tsukuba, Japan} % Location
    {Nov. 2019 - March. 2020} % Date(s)
    {
      \begin{cvitems} % Description(s) of tasks/responsibilities
        \item Responsable de la fusion des logiciels de contrôle entre:\\
          -- Le framework \entrypositionstyle{mc\_rtc} développé par le CNRS par l'équipe IDH au LIRMM (Montpellier) et l'AIST-JRL (Tsukuba)\\
          -- Le framework \entrypositionstyle{HMC} développé par le groupe HRG a l'AIST (Tsukuba)
        \item Soutien technique aux deux groupes, développement de démonstrations sur les robots afin d'assurer les contributions expérimentales des publications scientifiques ainsi que de répondre aux attentes de nos partenaires industriels.
      \end{cvitems}
    }

%---------------------------------------------------------
  \cventry
    {Ingénieur de recherche} % Job title
    {Interactive Digital Human -- LIRMM} % Organization
    {Montpellier, France} % Location
    {Oct. 2018 - Oct. 2019} % Date(s)
    {
      \begin{cvitems} % Description(s) of tasks/responsibilities
        \item \entrypositionstyle{H2020 COMANOID - Multi-Contact Collaborative Humanoids in Aircraft Manufacturing}\\
          \entrypositionstyle{Site}: \url{https://comanoid.cnrs.fr/}\\
          \entrypositionstyle{Rôle}: \textbf{Responable de l'implémentation et intégration des méthodes de localisation et cartographie} pour la démonstration finale du projet européen COMANOID. Cette demonstration, résultat de 4 ans d'efforts partagés entre quatre instituts de recherche (LIRMM, DLR, Sapienza Univeristy of Rome, INRIA Rennes) a permis de démontrer la viabilité des robots humanoïdes dans le cadre industriel réel de construction aéronautique, en présentant les capacités de locomotion et manipulation dans un espace contraint : marche et localisation (SLAM), montée d’escalier (MPC), manipulation (SLAM, registration, visual servoing).
        \item \entrypositionstyle{mc\_rtc}: \url{https://jrl-umi3218.github.io/}\\
        Développement et maintinen du framework de contrôle `mc\_rtc` utilisé lors de la démonstration ci-dessus, ainsi que par les étudiants et chercheurs du LIRMM, JRL, et leurs partenaires.
        \item Soutien technique aux étudiants et chercheurs et réalisation d'expériences sur les robots HRP-4 et BAZAR.
      \end{cvitems}
    }

%---------------------------------------------------------
  \cventry
    {These} % Job title
    {LIRMM, I3S, JRL} % Organization
    {France, Japon} % Location
    {Oct. 2014 - Nov. 2018} % Date(s)
    {
      \begin{cvitems} % Description(s) of tasks/responsibilities
        \item \entrypositionstyle{Directeurs}: Abderrahmane Kheddar, Andrew Comport 
        \item \entrypositionstyle{Projets}: RobotHow, H2020 COMANOID, DARPA Robotics Challenge
        \item Localisation d'un robot humanoïde et de son environnement exploitant l'état de l'art du SLAM visuel dense.
        \item Localisation d'objets par registration de modèles CAO avec la carte dense du SLAM.
        \item Adaptation en ligne de plans de locomotion multi-contacts générés hors ligne exploitant la localisation et cartographie du SLAM.
        \item Développement d'une méthode de calibration corps-complet. 
        \item Marche par commande prédictive de modèle (MPC), exploitant une fusion d'informations visuelles (SLAM) et de capteurs proprioceptifs (encoders, capteurs de force) permettant de réagir à des perturbations en générant continuellement une trajectoire de ZMP et les pas futurs assurant la stabilité du robot.
        \item \entrypositionstyle{\emph{DARPA Robotics Challenge (DRC)}}: Participation au sein de l'équipe AIST-NEDO. Classés 10/23 avec l'accomplissement de 6 des 8 tâches (conduite semi-autonome, ouverture de porte et d'une vanne, perçage d'un mur, raccordement d'un cable, traversée d'un terrain accidenté).
      \end{cvitems}
    }

%---------------------------------------------------------
  \cventry
    {Stagiaire} % Job title
    {Technische Universität München (TUM)} % Organization
    {Munich, Allemagne} % Location
    {2014 (6 months)} % Date(s)
    {
      \begin{cvitems} % Description(s) of tasks/responsibilities
        \item \entrypositionstyle{Superviseurs}: Jurgen Sturm et Daniel Cremers
        \item Application de réseaux de neurones convolutionels appliqués a la détection de fermeture de boucle du SLAM.
        \item Développement de l'architecture permettant l'utilisation de réseaux Siamois dans le framework open-source \emph{Caffe}.
      \end{cvitems}
    }

%---------------------------------------------------------
  \cventry
    {Projets universitaires} % Job title
    {Polytech Nice-Sophia-Antipolis, Trinity College Dublin} % Organization
    {France, Irelande, Allemagne} % Location
    {2014 (6 months)} % Date(s)
    {
      \begin{cvitems} % Description(s) of tasks/responsibilities
        \item Développement d'un moteur physique et de rendu (simulation de fluides, collisions entre objets rigides, collisions objets/fluide, raytracing)\\\url{https://github.com/arntanguy/PHEngine}.
        \item Développement d'un logiciel de fitting interactif de courbes spécialisées pour la recherche en spectrospie par microscope à effet tunnel\\\url{https://github.com/arntanguy/STS-simulator}. 
        \item Rendu photo-réaliste de cartes SLAM dans un Occulus Rift (projet dirigé par Andrew Comport).
        \item Développement d'un jeu de course 3D pour joueurs à handicap visuel\\\url{http://prdevint.polytech.unice.fr}. 
        \item Développement de jeux de réalitée augmentée.
      \end{cvitems}
    }

%---------------------------------------------------------
  \cventry
    {Lyceen, projet C++ d'apprentissage auto-didacte} % Job title
    {Fotowall} % Organization
    {Brest, France} % Location
    {2008-2011} % Date(s)
    {
      \begin{cvitems} % Description(s) of tasks/responsibilities
        \item \entrypositionstyle{Site}: \url{https://www.enricoros.com/opensource/fotowall/index.html}
        \item Développement autodidacte d'un logiciel C++ open-source de manipulation d'image Fotowall
        \item Collaboration à distance avec le développeur italien Enrico Ross
        \item Plus d'un million de telechargements (en 2017)
      \end{cvitems}
    }

\end{cventries}

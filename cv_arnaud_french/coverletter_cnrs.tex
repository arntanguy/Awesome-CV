%!TEX TS-program = xelatex
%!TEX encoding = UTF-8 Unicode
% Awesome CV LaTeX Template for Cover Letter
%
% This template has been downloaded from:
% https://github.com/posquit0/Awesome-CV
%
% Authors:
% Claud D. Park <posquit0.bj@gmail.com>
% Lars Richter <mail@ayeks.de>
%
% Template license:
% CC BY-SA 4.0 (https://creativecommons.org/licenses/by-sa/4.0/)
%


%-------------------------------------------------------------------------------
% CONFIGURATIONS
%-----------Aujourd'hui, fort de mon parcours, des compétences acquises tant en matières scientifiques, techniques, qu' humaines (...), je vous propose d'apporter mon soutien à la recherche au sein de l’équipe du département D’INFORMATIQUE, DE ROBOTIQUE ET DE MICROÉLECTRONIQUE du LIRMM à Montpellier et je me porte candidat au poste d'Expert en contrôle - Commande que vous proposez.--------------------------------------------------------------------
% A4 paper size by default, use 'letterpaper' for US letter
\documentclass[11pt, a4paper]{awesome-cv}
\usepackage{hyperref}


% Configure page margins with geometry
\geometry{left=1.4cm, top=.8cm, right=1.4cm, bottom=1.8cm, footskip=.5cm}

% Specify the location of the included fonts
\fontdir[fonts/]

% Color for highlights
% Awesome Colors: awesome-emerald, awesome-skyblue, awesome-red, awesome-pink, awesome-orange
%                 awesome-nephritis, awesome-concrete, awesome-darknight
\colorlet{awesome}{awesome-red}
% Uncomment if you would like to specify your own color
% \definecolor{awesome}{HTML}{CA63A8}

% Colors for text
% Uncomment if you would like to specify your own color
% \definecolor{darktext}{HTML}{414141}
% \definecolor{text}{HTML}{333333}
% \definecolor{graytext}{HTML}{5D5D5D}
% \definecolor{lighttext}{HTML}{999999}

% Set false if you don't want to highlight section with awesome color
\setbool{acvSectionColorHighlight}{true}

% If you would like to change the social information separator from a pipe (|) to something else
\renewcommand{\acvHeaderSocialSep}{\quad\textbar\quad}


%-------------------------------------------------------------------------------
%	PERSONAL INFORMATION
%	Comment any of the lines below if they are not required
%-------------------------------------------------------------------------------
% Available options: circle|rectangle,edge/noedge,left/right
\photo{profile}
\name{Arnaud}{Tanguy}
\position{Ingénieur de recherche{\enskip\cdotp\enskip}robotique}
\address{42-8, Bangbae-ro 15-gil, Seocho-gu, Seoul, 00681, Rep. of KOREA}
\address{Higashi 2-2-1, 203, 3050046 Tsukuba, Japan}

\mobile{(+81) 80-7538-5168}
\email{arn.tanguy@gmail.com}
% \homepage{https://github.com/arntanguy}
\github{arntanguy}
\linkedin{arnaud-tanguy}
% \gitlab{gitlab-id}
% \stackoverflow{SO-id}{SO-name}
% \twitter{@twit}
% \skype{skype-id}
% \reddit{reddit-id}
% \medium{madium-id}
\googlescholar{SeWUqycAAAAJ}{Google Scholar}
%% \firstname and \lastname will be used
% \googlescholar{googlescholar-id}{}
% \extrainfo{extra informations}

% \quote{``Ingénierie au service de la recherche et de ses applications."}



%-------------------------------------------------------------------------------
%	LETTER INFORMATION
%	All of the below lines must be filled out
%-------------------------------------------------------------------------------
% The company being applied to
\recipient
  {CNRS -- Candidature au poste Expert en contrôle-commande (Concours N° 36)}
  {Affectation: Laboratoire d'Informatique, de Robotique et de Microélectronique de Montpellier, MONTPELLIER}
% The date on the letter, default is the date of compilation
\letterdate{\today}
% The title of the letter
\lettertitle{}
% How the letter is opened
\letteropening{Madame, Monsieur,}
% How the letter is closed
\letterclosing{Veuillez agréer, Madame, Monsieur, l'expression de mes salutations respectueuses,}
% Any enclosures with the letter
\letterenclosure[Piece-jointe]{Curriculum Vitae, Lettres de recommandation}


%-------------------------------------------------------------------------------
\begin{document}

% Print the header with above personal informations
% Give optional argument to change alignment(C: center, L: left, R: right)
\makecvheader[R]

% Print the footer with 3 arguments(<left>, <center>, <right>)
% Leave any of these blank if they are not needed
\makecvfooter
  {\today}
  {Arnaud Tanguy~~~·~~~Lettre de motivation}
  {}

% Print the title with above letter informations
\makelettertitle

%-------------------------------------------------------------------------------
%	LETTER CONTENT
%-------------------------------------------------------------------------------
\begin{cvletter}

Passionné d'informatique depuis le collège, mes premiers pas en programmation furent autodidactes. Après une période de découvertes variées explorant la diversité de l'écosystème informatique, c'est par le biais du développement du logiciel \entrypositionstyle{C++} de traitement d'images \entrypositionstyle{Fotowall}, durant mes années de collège/lycée, au travers d'une collaboration à distance, que j'ai réellement fait mes premiers pas dans le monde du développement logiciel. Cette collaboration m'a permis d'étendre mes compétences en programmation, et de découvrir les rudiments du développement collaboratif et de ses outils. Tout naturellement mon parcours s'est orienté vers les écoles d'ingénieur informatique. Les deux années de classes préparatoires MPSI m'ont fait entrevoir les possibilités offertes par les mathématiques et la physique, mises en pratique en école d'ingénieur au travers de multiples projets. Les stages de dernière année m'ont introduits aux domaines de la vision et de la commande et planification robotique et donné l'envie d'approfondir leur étude au travers d'une thèse sur le thème du \entrypositionstyle{``SLAM visuel pour la localisation et la commande en boucle fermée de robots humanoïdes"}.

De par sa nature, à l'intersection des domaines de vision et commande robotique, et son ambition à rapprocher l'état de l'art des deux domaines, cette thèse m'a mis face à d'uniques défis et permis d'explorer en profondeur leurs problématiques scientifiques. Elle m'a d'autre part m'a fait prendre conscience du rôle indispensable de l'ingénierie et de l'intégration continue de logiciels au bon déroulement des travaux de recherche. Il est clair que ceux-ci n'auraient pas pu voir le jour sans construire sur les efforts de mes prédécesseurs à produire une implémentation rigoureuse, performante et flexible de l'état de l'art.

Ma carrière s'est poursuivie au \entrypositionstyle{CNRS}, avec un poste d'ingénieur contractuel au \entrypositionstyle{LIRMM}, me permettant ainsi de contribuer aux activités de recherche, notamment par la poursuite du développement du framework de contrôle open-source \entrypositionstyle{mc\_rtc}. Celui-ci a pour ambition de fournir une implémentation des algorithmes de pointe en commande de robots, ainsi que les outils essentiels au développement de nouvelles méthodes. Ces travaux ainsi que les efforts de recherche menés au cours des quatre dernières années ont aboutit à la réalisation du démonstrateur final du projet européen \entrypositionstyle{H2020 COMANOID} quant à l'utilisation d'un robot humanoïde de manufacture dans un contexte industriel réel~: une première mondiale~! Cette expérience m'a conduit à mener pour l'\entrypositionstyle{AIST} un projet de fusion entre {\tt mc\_rtc} et leurs méthodes de contrôle, afin de combiner le savoir-faire respectif du CNRS et de l'AIST et d'aboutir à un logiciel de commande unique qui sera enrichi par tous les chercheurs du CNRS, de l'AIST et évidemment d'au-delà. Par exemple, le logiciel est actuellement enrichi par TU/e (Université de Eindhoven), TUM (Université de Munich) et l'EPFL de Lausanne qui apportent respectivement une expertise particulière (TU/e, équipe de Henk Nijmeijer, pour la rigueur des preuves de stabilités sous des tâches d'impacts; à TUM l'équipe de Sami Haddadin pour les observateurs de contacts performants, et à l'EPFL l'équipe de Aude Billard pour planification des tâches par des systèmes dynamiques, réglages des gains par apprentissage, etc.).  

Au cours de mon parcours académique, j'ai eu lieu d'interagir avec de nombreuses plateformes robotiques (au LIRMM, à I3S comme à l'AIST). Par ailleurs, j'ai également eu l'opportunité d'étudier et travailler au sein de multiples universités et laboratoires, en France comme a l'étranger, en interaction avec des collègues d'origine et de cultures variées.

Aujourd'hui, fort de mon parcours, des compétences acquises tant en matières scientifiques, techniques, qu'humaines, je vous propose d'apporter mon soutien au département robotique du \entrypositionstyle{LIRMM} à Montpellier, dont je connais très bien les divers enjeux scientifiques et autres robots. Je me porte donc candidat au poste d'\entrypositionstyle{expert en contrôle-commande} proposé par le CNRS. 

Je m'engage fortement à apporter une solution intégrée pérenne aux développements logiciels de commande de tous les robots du département robotique et un soutien énergique à la recherche, aux projets européens, nationaux et internationaux ainsi que les projets industriels du département. Je m'engage également à converger vers un framework soutenu par l'ensemble des chercheurs, incluant les problématiques de perception et de contrôle. 

\end{cvletter}


%-------------------------------------------------------------------------------
% Print the signature and enclosures with above letter informations
\makeletterclosing

\end{document}

%!TEX TS-program = xelatex
%!TEX encoding = UTF-8 Unicode
% Awesome CV LaTeX Template for Cover Letter
%
% This template has been downloaded from:
% https://github.com/posquit0/Awesome-CV
%
% Authors:
% Claud D. Park <posquit0.bj@gmail.com>
% Lars Richter <mail@ayeks.de>
%
% Template license:
% CC BY-SA 4.0 (https://creativecommons.org/licenses/by-sa/4.0/)
%


%-------------------------------------------------------------------------------
% CONFIGURATIONS
%-----------Aujourd'hui, fort de mon parcours, des compétences acquises tant en matières scientifiques, techniques, qu' humaines (...), je vous propose d'apporter mon soutien à la recherche au sein de l’équipe du département D’INFORMATIQUE, DE ROBOTIQUE ET DE MICROÉLECTRONIQUE du LIRMM à Montpellier et je me porte candidat au poste d'Expert en contrôle - Commande que vous proposez.--------------------------------------------------------------------
% A4 paper size by default, use 'letterpaper' for US letter
\documentclass[11pt, a4paper]{awesome-cv}
\usepackage{hyperref}


% Configure page margins with geometry
\geometry{left=1.4cm, top=.8cm, right=1.4cm, bottom=1.8cm, footskip=.5cm}

% Specify the location of the included fonts
\fontdir[fonts/]

% Color for highlights
% Awesome Colors: awesome-emerald, awesome-skyblue, awesome-red, awesome-pink, awesome-orange
%                 awesome-nephritis, awesome-concrete, awesome-darknight
\colorlet{awesome}{awesome-red}
% Uncomment if you would like to specify your own color
% \definecolor{awesome}{HTML}{CA63A8}

% Colors for text
% Uncomment if you would like to specify your own color
% \definecolor{darktext}{HTML}{414141}
% \definecolor{text}{HTML}{333333}
% \definecolor{graytext}{HTML}{5D5D5D}
% \definecolor{lighttext}{HTML}{999999}

% Set false if you don't want to highlight section with awesome color
\setbool{acvSectionColorHighlight}{true}

% If you would like to change the social information separator from a pipe (|) to something else
\renewcommand{\acvHeaderSocialSep}{\quad\textbar\quad}


%-------------------------------------------------------------------------------
%	PERSONAL INFORMATION
%	Comment any of the lines below if they are not required
%-------------------------------------------------------------------------------
% Available options: circle|rectangle,edge/noedge,left/right
\photo{profile}
\name{Arnaud}{Tanguy}
\position{Ingénieur de recherche{\enskip\cdotp\enskip}robotique}
\address{42-8, Bangbae-ro 15-gil, Seocho-gu, Seoul, 00681, Rep. of KOREA}
\address{Higashi 2-2-1, 203, 3050046 Tsukuba, Japan}

\mobile{(+81) 80-7538-5168}
\email{arn.tanguy@gmail.com}
% \homepage{https://github.com/arntanguy}
\github{arntanguy}
\linkedin{arnaud-tanguy}
% \gitlab{gitlab-id}
% \stackoverflow{SO-id}{SO-name}
% \twitter{@twit}
% \skype{skype-id}
% \reddit{reddit-id}
% \medium{madium-id}
\googlescholar{SeWUqycAAAAJ}{Google Scholar}
%% \firstname and \lastname will be used
% \googlescholar{googlescholar-id}{}
% \extrainfo{extra informations}

% \quote{``Ingénierie au service de la recherche et de ses applications."}



%-------------------------------------------------------------------------------
%	LETTER INFORMATION
%	All of the below lines must be filled out
%-------------------------------------------------------------------------------
% The company being applied to
\recipient
  {CNRS -- Candidature au poste Expert en contrôle-commande (Concours N° 36)}
  {Affectation: Laboratoire d'Informatique, de Robotique et de Microélectronique de Montpellier, MONTPELLIER}
% The date on the letter, default is the date of compilation
\letterdate{\today}
% The title of the letter
\lettertitle{}
% How the letter is opened
\letteropening{Madame, Monsieur,}
% How the letter is closed
\letterclosing{Veuillez agréer, Madame, Monsieur, l'expression de mes salutations distinguées,}
% Any enclosures with the letter
\letterenclosure[Piece-jointe]{Curriculum Vitae}


%-------------------------------------------------------------------------------
\begin{document}

% Print the header with above personal informations
% Give optional argument to change alignment(C: center, L: left, R: right)
\makecvheader[R]

% Print the footer with 3 arguments(<left>, <center>, <right>)
% Leave any of these blank if they are not needed
\makecvfooter
  {\today}
  {Arnaud Tanguy~~~·~~~Cover Letter}
  {}

% Print the title with above letter informations
\makelettertitle

%-------------------------------------------------------------------------------
%	LETTER CONTENT
%-------------------------------------------------------------------------------
\begin{cvletter}

  Passionné d'informatique depuis le collège, mes premiers pas en programmation furent autodidactes. Après une période de découvertes variées explorant la diversité de l'écosystème informatique, c'est par le biais du développement du logiciel C++ de traitement d'images Fotowall\footnote{\url{https://www.enricoros.com/opensource/fotowall/}} durant mes années de collège/lycée, et ce en collaboration avec le développeur italien Enrico Ross, que j'ai réellement fait mes premiers pas dans le monde du développement logiciel. Cette collaboration m'a permis d'étendre mes compétences en programmation, et de découvrir les rudiments du développement collaboratif et de ses outils. Tout naturellement mon parcours s'est orienté vers les écoles d'ingenieur informatique. Les deux années de classes préparatoires MPSI m'ont fait entrevoir les possibilités offertes par les mathématiques et la physique. Les trois ans d'école d'ingénieur qui ont suivit m'on permis de les mettre en pratique, au travers de multiples projets tels que le développement d'un moteur physique et de rendu 3D, un programme scientifique de fitting de courbes spécialisé pour la recherche en microscopie à effet tunnel. Les stages de dernière année m'ont permis d'appréhender l'état de l'art en terme de localisation et mapping (SLAM) au travers d'un projet de réalité augmentée avec Dr. Andrew Comport; ainsi que le deep-learning au travers d'un projet de reconnaissance de lieux par réseaux de neurones convolutionnels (supervisé par Dr. Jürgen Sturm).

  C'est, armé de ces compétences et de cette passion de longue date que mon parcours s'est poursuivi par un doctorat sur le thème du \emph{``SLAM visuel pour la localisation et la commande en boucle fermée de robots humanoïdes"}. Ce projet m'a permis de continuer à développer mes connaissances en vision par ordinateur, ainsi que de découvrir le monde de la robotique et ses problèmes, tant complexes que fascinants. Problèmes scientifiques et fondamentaux d'une part, nécessitant d'autre part des efforts d'ingénierie non négligeables pour être menés a bien. Conception, création et entretien de systèmes robotiques, développement des multiples couches logicielles permettant leur contrôle. De par sa nature a l'intersection des domaines de vision et robotique, et de par son ambition à rapprocher l'état de l'art des deux domaines, cette thèse m'a mis face à d'uniques défis, tant en terme de recherche que d'ingénierie. Il est clair que ces travaux n'auraient pu voir le jour sans construire sur les efforts de mes pairs à produire une implémentation rigoureuse, performante et flexible de l'état de l'art.

  C'est dans l'optique d'appliquer mes compétences multiples en ingénierie logicielle, robotique et vision que j'ai poursuivi ma carrière en tant qu'ingénieur de recherche. Une première année au CNRS affecté au LIRMM m'a donné l'occasion de mettre ces compétences au profit des chercheurs du laboratoire, notemment par la poursuite du développement du framework de contrôle open-source \entrypositionstyle{mc\_rtc}\footnote{\url{https://jrl-umi3218.github.io/mc_rtc/}} visant à mettre a disposition des chercheurs les algorithmes et outils essentiels a leurs travaux. Au-delà des multiples accomplissements scientifiques, cette année fut marquée par la réalisation de la démonstration finale du projet européen \entrypositionstyle{H2020 COMANOID}\footnote{\url{https://comanoid.cnrs.fr/}}, démontrant la capacité d'un robot humanoïde à évoluer et réaliser des tâches de construction dans le contexte industriel réel de construction aéronautique: une première mondiale!

  Toujours dans cette même optique, j'ai poursuivi mon parcours par un projet de fusion entre les logiciels de contrôle développés par l'équipe Humanoid Research Group de l'AIST (Tsukuba, Japon) et ceux développés par le CNRS au LIRMM ainsi qu'au JRL. Ce projet est né du constat que les problématiques complexes actuelles en robotique ne peuvent être surmontées que par le biais de collaborations, non seulement scientifiques, mais aussi techniques. Cette volonté d'intégration et de collaboration ne se limite pas au strict domaine de la robotique humanoïde, mais vise à être suffisamment générale pour garantir la pérénnité de la recherche en robotique au sens large.

  Au cours de mon parcours académique, j'ai eu lieu d'interagir avec de nombreuses plateformes robotiques, tant celles du LIRMM  (HRP-4, BAZAR, Kukka, robots a câbles, ...) que celles de l'AIST (HRP-2Kai, HRP5P, Sawyer, Franka, ...). Par ailleurs, j'ai également eu l'opportunité d'étudier et travailler au sein de multiples universités et laboratoires, en france comme a l'étranger, en interaction avec des collègues d'origines et de cultures variées.

  Aujourd'hui, fort de mon parcours, des compétences acquises tant en matières scientifiques, techniques, qu' humaines, je vous propose d'apporter mon soutien à la recherche au sein de l’équipe du département \entrypositionstyle{D’INFORMATIQUE, DE ROBOTIQUE ET DE MICROÉLECTRONIQUE du LIRMM} à Montpellier, et je me porte candidat au poste d'\entrypositionstyle{Expert en contrôle - Commande} que vous proposez.

\end{cvletter}


%-------------------------------------------------------------------------------
% Print the signature and enclosures with above letter informations
\makeletterclosing

\end{document}

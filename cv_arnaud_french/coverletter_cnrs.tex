%!TEX TS-program = xelatex
%!TEX encoding = UTF-8 Unicode
% Awesome CV LaTeX Template for Cover Letter
%
% This template has been downloaded from:
% https://github.com/posquit0/Awesome-CV
%
% Authors:
% Claud D. Park <posquit0.bj@gmail.com>
% Lars Richter <mail@ayeks.de>
%
% Template license:
% CC BY-SA 4.0 (https://creativecommons.org/licenses/by-sa/4.0/)
%


%-------------------------------------------------------------------------------
% CONFIGURATIONS
%-------------------------------------------------------------------------------
% A4 paper size by default, use 'letterpaper' for US letter
\documentclass[11pt, a4paper]{awesome-cv}
\usepackage{hyperref}


% Configure page margins with geometry
\geometry{left=1.4cm, top=.8cm, right=1.4cm, bottom=1.8cm, footskip=.5cm}

% Specify the location of the included fonts
\fontdir[fonts/]

% Color for highlights
% Awesome Colors: awesome-emerald, awesome-skyblue, awesome-red, awesome-pink, awesome-orange
%                 awesome-nephritis, awesome-concrete, awesome-darknight
\colorlet{awesome}{awesome-red}
% Uncomment if you would like to specify your own color
% \definecolor{awesome}{HTML}{CA63A8}

% Colors for text
% Uncomment if you would like to specify your own color
% \definecolor{darktext}{HTML}{414141}
% \definecolor{text}{HTML}{333333}
% \definecolor{graytext}{HTML}{5D5D5D}
% \definecolor{lighttext}{HTML}{999999}

% Set false if you don't want to highlight section with awesome color
\setbool{acvSectionColorHighlight}{true}

% If you would like to change the social information separator from a pipe (|) to something else
\renewcommand{\acvHeaderSocialSep}{\quad\textbar\quad}


%-------------------------------------------------------------------------------
%	PERSONAL INFORMATION
%	Comment any of the lines below if they are not required
%-------------------------------------------------------------------------------
% Available options: circle|rectangle,edge/noedge,left/right
\photo{profile}
\name{Arnaud}{Tanguy}
\position{Ingénieur de recherche{\enskip\cdotp\enskip}robotique}
\address{42-8, Bangbae-ro 15-gil, Seocho-gu, Seoul, 00681, Rep. of KOREA}
\address{Higashi 2-2-1, 203, 3050046 Tsukuba, Japan}

\mobile{(+81) 80-7538-5168}
\email{arn.tanguy@gmail.com}
% \homepage{https://github.com/arntanguy}
\github{arntanguy}
\linkedin{arnaud-tanguy}
% \gitlab{gitlab-id}
% \stackoverflow{SO-id}{SO-name}
% \twitter{@twit}
% \skype{skype-id}
% \reddit{reddit-id}
% \medium{madium-id}
\googlescholar{SeWUqycAAAAJ}{Google Scholar}
%% \firstname and \lastname will be used
% \googlescholar{googlescholar-id}{}
% \extrainfo{extra informations}

% \quote{``Ingénierie au service de la recherche et de ses applications."}



%-------------------------------------------------------------------------------
%	LETTER INFORMATION
%	All of the below lines must be filled out
%-------------------------------------------------------------------------------
% The company being applied to
\recipient
  {CNRS -- Candidature au poste Expert en contrôle-commande (Concours N° 36)}
  {Affectation: Laboratoire d'Informatique, de Robotique et de Microélectronique de Montpellier, MONTPELLIER}
% The date on the letter, default is the date of compilation
\letterdate{\today}
% The title of the letter
\lettertitle{}
% How the letter is opened
\letteropening{Madame, Monsieur,}
% How the letter is closed
\letterclosing{Veuillez agréer, Madame, Monsieur, l'expression de mes salutations distinguées,}
% Any enclosures with the letter
\letterenclosure[Piece-jointe]{Curriculum Vitae}


%-------------------------------------------------------------------------------
\begin{document}

% Print the header with above personal informations
% Give optional argument to change alignment(C: center, L: left, R: right)
\makecvheader[R]

% Print the footer with 3 arguments(<left>, <center>, <right>)
% Leave any of these blank if they are not needed
\makecvfooter
  {\today}
  {Arnaud Tanguy~~~·~~~Cover Letter}
  {}

% Print the title with above letter informations
\makelettertitle

%-------------------------------------------------------------------------------
%	LETTER CONTENT
%-------------------------------------------------------------------------------
\begin{cvletter}

  Passionné d'informatique depuis le collège, mes premiers pas en programmation furent autodidactes. Après une période de découverte variée explorant la diversité de l'écosystème informatique, de Windows à Linux, de programmation web au C++, de simples applications à des projets de plus en plus complexes. C'est par le biais du développement du logiciel de traitement d'images Fotowall\footnote{\url{https://www.enricoros.com/opensource/fotowall/}} durant mes années de collège/lycée, et ce en collaboration avec un développer italien que j'ai réellement fait mes premiers pas dans le monde du développement logiciel. Cette collaboration m'a permis d'étendre mes compétences en programmation, et de découvrir les rudiments du développement collaboratif et de ses outils. C'est tout naturellement que mon parcours s'est orienté vers les écoles d'ingenieurs en informatique. Les deux années de classes préparatoires MPSI m'ont fait entrevoir les possibilités offertes par les mathématiques et la physique, possibilités que les trois ans d'école d'ingenieur m'ont offert l'opportunité de mettre en pratique, au travers de multiples projets tels que le développement d'un moteur physique\footnote{\url{https://github.com/arntanguy/PHEngine}} et de rendu 3D pour jeux videos, un programme scientifique de fitting de courbes spécialisé pour la recherche en microscopie a effet tunnel\footnote{\url{https://github.com/arntanguy/STS-simulator}}. Les stages de dernière année m'ont fait découvrir l'état de l'art en terme de localisation et mapping (SLAM) au travers d'un projet de réalité augmentée avec Dr. Andrew Comport; ainsi que le deep-learning au travers d'un projet de reconnaissance de lieux par réseaux de neurones convolutionnels (supervisé par Dr. Jürgen Sturm).

  C'est armé de ces compétences et passion de longue date que mon parcours s'est poursuivi par un doctorat sur le thème du \emph{``SLAM visuel pour la localisation et la commande en boucle fermée de robots humanoïdes"}. Ce projet m'a permis de continuer à développer mes compétences en vision par ordinateur, ainsi que de découvrir le monde de la robotique et ses problèmes, tant complexes que fascinants. Problèmes scientifiques et fondamentaux d'une part, nécessitant d'autre part des efforts d'ingénierie conséquents pour être menés a bien. Conception, création et entretien de systèmes robotiques, développement des multiples couches logicielles permettant leur contrôle, allant du code bas niveau de ces robots jusqu'à l'implémentation haut niveau de l'état de l'art du domaine, et poussant au-delà par les travaux de recherche. De par sa nature a l'intersection des domaines de vision et robotique, et de par son ambition à rapprocher l'état de l'art des deux domaines, cette thèse m'a présenté d'uniques défis, tant en terme de recherche que d'ingénierie; sa réalisation pratique ne pouvant être accomplie que par l'implémentation rigoureuse des methodes proposées sur des robots humanoïdes réels. Il est clair que ces travaux n'auraient pu voir le jour sans construire sur les efforts de mes pairs a produire une implementation rigoureuse, performante et flexible de l'état de l'art.

  C'est dans l'optique d'appliquer mes compétences multiples en ingénierie logicielle, robotique et vision que j'ai poursuivi ma carière en tant qu'ingénieur de recherche. Une première année au CNRS affecté au LIRMM m'a permis de mettre ces compétences au profit de la recherche et des chercheurs du laboratoire, par la poursuite du développement du framework de contrôle open-source mc\_rtc\footnote{\url{https://jrl-umi3218.github.io/mc_rtc/}} visant a rendre l'état de l'art de la robotique humanoïde accessible a tous, et à fournir les outils nécessaires aux chercheurs pour explorer efficacement de nouvelles méthodes. Au-delà des multiples accomplissements scientifiques, cette année fut marquée par l'accomplissement de la démonstration finale du projet européen H2020 COMANOID\footnote{\url{https://comanoid.cnrs.fr/}}, démontrant la capacité d'un robot humanoïde a évoluer et réaliser des tâches de construction dans le contexte industriel réel de la construction aéronautique, une première mondiale!

  Toujours dans cette même optique, j'ai poursuivi mon parcours par un projet de fusion entre les logiciels de contrôle développés par l'équipe Humanoid Research Group de l'Advanced Institute of Science and Technology (Tsukuba, Japan) et ceux développés par le CNRS au LIRMM et au Joint Robotics Laboratory. Ce projet est né de la réalisation que les problématiques complexes actuelles en robotique ne peuvent être surmontés que par le biais de collaborations, non seulement scientifiques, mais aussi techniques. Cette volonté d'intégration et de collaboration ne se limite pas au strict domaine de la robotique humanoïde, mais vise à être suffisament génerale pour guarantir la pérénité de la recherche en robotique au sens large.

  De part mes compétences techniques, ainsi que mon experience, tant au LIRMM en interaction avec ses multiples plateformes robotiques (HRP-4, BAZAR, Kukka, robots a cables, etc.), qu'a l'AIST au contact d'HRP-2Kai, HRP5P, Sawyer, Franka (etc.), ainsi que dans les autres laboratoires par lesquels je suis passé (I3S, TUM), je pense pouvoir apporter les compétences nécessaires pour soutenir la recherche au sein du département robotique du LIRMM. 

\end{cvletter}


%-------------------------------------------------------------------------------
% Print the signature and enclosures with above letter informations
\makeletterclosing

\end{document}
